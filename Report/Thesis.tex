\documentclass[10pt,a5paper]{scrartcl}
\usepackage[utf8]{inputenc}
\usepackage[english,russian]{babel}
\begin{document}
    \begin{center}
        \textbf{Электроника стенда по изучению сцинтилляционных кристаллов}
        \bigbreak
        А. А. Андреев \\
        Институт ядерной физики им. Г. И. Будкера СО РАН г. Новосибирск
    \end{center}

    Детекторы ионизирующего излучения --- это одни из наиболее важных элементов практически любой современной экспериментальной установки в физике высоких энергий. В Институте ядерной физики СО РАН реализуется проект по выращиванию неорганических сцинтилляционных кристаллов, которые являются неотъемлемой частью таких детекторов. Для исследования характеристик изготавливаемых кристаллов был разработан стенд на основе системы на кристалле Xilinx Zynq 7000. \par
    Целью данной работы является разработка прошивки для Xilinx Zynq 7000, которая включает в себя следующие модули:\par
    1) дизайн программируемой логики;\par
    2) образ операционной системы;\par
    3) серверная часть.\par
    Основной задачей программируемой логики является десериализация оцифрованных данных, их обработка и передача в процессорную часть. Данный модуль разрабатывается на языке описания аппаратуры интегральных схем VHDL. На процессоре будет запущен образ дистрибутива Linux Petalinux, специально собранный под разработанный дизайн логики системы на кристалле Zynq 7000. Для отображения конечных результатов и управления стендом будет работать веб-сервер, разрабатываемый на языке Python с использованием фреймворка Django. Для его корректной работы в операционную систему добавлены необходимые пакеты и интерпретатор языка Python. Разработка вебинтерфейса производится с помощью языков HTML, CSS и JavaScript.\par
    В результате данной работы, будет завершена разработка электроники стенда по исследованию сцинтилляционных кристаллов. Обработка данных будет осуществляться с применением программируемой логики, а отображение информации и управление стендом будет происходить через удобный веб-интерфейс.\par
\bigbreak
\begin{center}
    Научный руководитель --- В. В. Жуланов 
\end{center} 
\end{document}

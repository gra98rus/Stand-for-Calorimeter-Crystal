Для создания программируемой логики внути СнК и реализации её взаимодействия с процессором проектируется дизайн системы на кристалле. Соответственно, он делится на две части: программируемая логика и процессорная система, для каждой из которых разрабатывается свой поддизайн. Важной задачей является осуществление их связи. В данной работе используется следующие типы взаимодействия:
\begin{itemize}
    \item чтение/запись данных напрямую в блок памяти через выделенный порт;
    \item чтение/запись регистров.
\end{itemize}\par
Как уже было сазано выше, в системе на кристалле Zynq-7000 в качестве управляющего протокола используется шина AXI, которая имеет 3 интерфейса:
\begin{itemize}
    \item AXI4 --- для высокопроизводительной работы с устройствами, отображаемых на адресное пространство памяти;
    \item AXI4-Lite --- для простой, низкоскоростной связи с переферийными устройствами, не имеющих высоких требований к пропускной способности(например, для регистров управления и состояния);
    \item AXI4-Stream --- для высокоскоростной потоковой передачи данных.
\end{itemize}\par
Разработка производилась в среде программирования Xilinx Vivado Design Suite на языке описания программной аппаратуры интегральных схем VHDL.

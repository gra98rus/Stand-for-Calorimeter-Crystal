Детекторы ионизирующего излучения --- это одни из наиболее важных элементов практически любой современной экспериментальной установки в области физики высоких энергий. В институте ядерной физики СО РАН реализуется проект по выращиванию неорганических сцинтилляционных кристаллов, которые являются неотъемлемой частью таких детекторов. Сцинтилляторы --- это вещества, способные излучать фотоны при поглощении ионизирующего излучения.\par
Для проверки характеристик и качества изготавливаемых сцинтилляционных кристаллов ведётся разработка специального стенда. Данный стенд имеет довольно сложное устройство, о нём будет рассказано подробнее в разделе "Установка стенда по исследованию сцинтилляционных кристаллов". Упраляющим компонентом стенда является система на кристалле(СнК) Xilinx Zynq-7000, являющаяся объединением процессора и программируемой логической интегральной схемы. Оператор сможет через порт Ethernet подключиться к веб-серверу, запущенному на СнК, через который будет производиться управление стендом и визулизация данных. Оценка параметров исследуемых сцинтилляционных кристаллов производится путём настройки временных характеристик формирователей входных сигналов.\par
Ранее для взаимодействия со стендом было начато создание интерфейса --- веб-сервера, запускаемого непосредственно на СнК, доступ к которому оператор получал через порт Ethernet. Также была частично реализована программируемая логика, подробнее она будет описана в соответствующей главе.\par

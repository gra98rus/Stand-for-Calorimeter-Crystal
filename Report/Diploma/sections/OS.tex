В качестве операционной системы был выбран легковесный дистрибутив ОС Linux PetaLinux \parencite{PetaLinuxRG}. Данное решение предоставляется компанией Xilinx и позволяет упаковывать в образ программной платформы дизайн системы на кристалле вместе с операционной системой. Конфигурирование образа PetaLinux реализуется с помощью пакета PetaLinux Tools, который содержит в себе Yocto Extensible SDK необходимый для возможности изменения файловой системы ОС. Среди преимуществ данного решения можно отметить наличие подробной документации \parencite{PetaLinuxWT, PetaLinuxCLRG}, что делает его более удобным по сравнению с другими вариантами, такими как OpenBricks или Buildroot.\par
Среди ограничений, накладываемых на образ операционной системы можно выделить:\par
\begin{itemize}
    \item размер образа не должен превышать объёма энегронезависимой памяти, устанновленного на стенде (64 МБ);
    \item система должна работать автономно.
\end{itemize}\par
Для корректной работы системы на стенде к ней предъявляются следующие требования:\par
\begin{itemize}
    \item настроенный сетевой интерфейс для подключения с ПК оператора;
    \item наличие всех необходимых библиотек в файловой системе для работы сервера.
\end{itemize}\par
Для этого в файловую систему были добавлены такие компоненты как интерпритатор языка python, пакеты для работы веб-сервера и, непосредственно, сам сервер. Также был написана утилита для настройки сетевого интерфеса, а также скрипт для автоматической установки необходимых пакетов и запуска веб-сервера при старте системы.

Первый сцинтилляционный детектор назывался спинтарископом и был изобретён Уильямом Круксом в 1903 году. Главной его частью был небольшой экран, покрытый сульфидом цинка (ZnS). При попадании на него заряженных $\alpha$-частиц возникает слабая световая вспышка - сцинтилляция, которую можно наблюдать в микроскоп или даже адаптированным к темноте невооружёным глазом.\par
Замена человеческого зрения на высокочувствительный фотоумножитель, а также использование усовершенствованных сцинтилляторов позволили данному типу 

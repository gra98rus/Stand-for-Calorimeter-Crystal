На рисунке() представлена блок-схема стенда.\par
\begin{figure}[ht]
    \centering
    \includegraphics[width=1\linewidth]{board_scheme.png}
    \caption{Блок-схема стенда}
    \label{fig:mpr}
\end{figure}
Как было сказано выше, стенд имеет 3 входных канала: основной и 2 вспомогательных. На каждом из них предусмотрен усилитель, сигнал с которого в набор формирователей, определяющих время формирования сигнала. Далее через промежуточный буфер с дифференциальным выходом сигнал поступает в 14-битный АЦП, где происходит его конвертация в цифровой вид. Оцифровка происходит на тактовой частоте 125 МГц, выдаваемой модулем фазовой автоподстройки частоты (ФАПЧ), реализованным в системе на кристалле. Цифровые данные в последовательно упакованном формате передаются в СнК, где проводится их обработка. В первую очередь, происходит обратная конвертация из последовательности бит в число (десериализация), после чего, при срабатывании триггерной системы, данные из блока буферов отправляются в процессор для последующей обработки. Для связи с компьютером используется протокол Ethernet.\par

На рисунке () представлена фотография стенда с выделением основных блоков:\par
\begin{enumerate}
    \item Блок питаний;
    \item Система на кристалле Zynq-7000 с необходимой переферией;
    \item 4-х канальный АЦП;
    \item Формирователи основного и вспомогательных каналов
    \item Усилители сигналов;
    \item Входные разъёмы. 
\end{enumerate}

\begin{figure}[ht]
    \centering
    \includegraphics[width=1\linewidth]{board.jpg}
    \caption{Фотография стенда}
    \label{fig:mpr}
\end{figure}

Далее будут рассмотренны подробнее особенности устройства некоторых частей описанной системы.\par
\textbf{Основной канал}\par
Основной канал имеет 2 набора формирователей с различными временами формирования: 0.1, 0.2, 0.5 и 1, 2, 5 мкс соответственно. Данные временные значения формирователей подобраны на основе анализа свойств сцинтилляционных кристаллов, а так же имеющегося опыта работы с ними. Такие времена способны обеспечить корректную работу с большим набором кристаллов: от со сравнительно малыми временами высвечивания до больших.\par
После каждого формирователя установлены электронные ключи, с помощью которых можно подключить выход одной секции из набора через буфер ко входу АЦП. На основном канале, таким образом, имеется возможность подключить одновременно 2 формирователя из диапазона 0.1 - 0.5 и 1 - 5 мкс соответственно. Оцифрованные данные непрерывно записываются в кольцевой буфер, откуда они могут быть выгружены для обработки и сохранения при поступлении команды о полезном событии от триггерной системы.\par
\textbf{Вспомогательные каналы}\par
Вспомогательные каналы служат источниками дополнительных сигналов, необходимых для правильной работы триггерной системы. Устройство вспомогательных каналов аналогично основному во всём, кроме:\par
\begin{itemize}
    \item каждый из них содержит только по одному набору формирователей с тремя секциями с временами 0.1, 0.2, и 0.5 мкс;
    \item к аналогово-цифровому преобразователю может быть подключен выход только одной секции формирования канала.
\end{itemize}\par
\textbf{Триггерная система}\par
Триггерная система выполняет задачу формирования сигнала, означающего возникновение полезного события, при котором данные из кольцевого буфера необходимо выгрузить для последующей обработки. Система может работать в двух режимах: принудительный старт и срабатывание по порогу. В первом случае триггерная система вырабатывает сигнал при получении команды от оператора. Во втором случае триггер срабатывает при превышении текущими цифровыми значениями основного и/или некоторых вспомогательных каналов заданных оператором порогов.

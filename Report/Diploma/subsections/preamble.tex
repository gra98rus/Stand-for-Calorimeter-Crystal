\usepackage[T2A]{fontenc}
\usepackage[utf8]{inputenc}%включаем свою кодировку: koi8-r или utf8 в UNIX, cp1251 в Windows
\usepackage[english,russian]{babel}%используем русский и английский языки с переносами
\usepackage{amssymb,amsfonts,amsmath,mathtext,enumerate,float} %подключаем нужные пакеты расширений
\usepackage{graphicx}
\usepackage{cmap}
\usepackage{verbatim}
\graphicspath{{images/}}%путь к рисункам
\bibliographystyle{gost-numeric.bbx}
\usepackage[parentracker=true,
            backend=biber,
            bibencoding=utf8,
            style=numeric-comp,
            language=auto,
%            autolang=other,
            citestyle=gost-numeric,
%            defernumbers=true,
            bibstyle=gost-numeric,
            sorting=none
]{biblatex}
\addbibresource{sections/bibliography.bib}
\usepackage{geometry} % Меняем поля страницы
\geometry{left=3cm}% левое поле
\geometry{right=2cm}% правое поле
\geometry{top=2cm}% верхнее поле
\geometry{bottom=2cm}% нижнее поле
\tolerance=3000
\geometry{marginparwidth=2.5cm} % Ширина заметок на полях.
\usepackage{biblatex}

%\usepackage{cite}

\makeatletter
\renewcommand{\@biblabel}[1]{#1.} % Заменяем библиографию с квадратных скобок на точку:
\makeatother

\usepackage{float}
\usepackage{xcolor}
\usepackage{marginnote}
\reversemarginpar % Перенос заметок с правого поля на левое.
\usepackage{hyperref}
\usepackage{tabularx}

% Полезные типы колонок: колонки с выравниваем влево/по центру/справа 
% фиксированной ширины (как p{}, но не justify).
\newcolumntype{L}[1]{>{\raggedright\let\newline\\\arraybackslash\hspace{0pt}}m{#1}}
\newcolumntype{C}[1]{>{\centering\let\newline\\\arraybackslash\hspace{0pt}}m{#1}}
\newcolumntype{R}[1]{>{\raggedleft\let\newline\\\arraybackslash\hspace{0pt}}m{#1}}

% Колонки для tabularx
\newcolumntype{E}[1]{>{\hsize=#1\hsize\raggedright\arraybackslash}X}%
\newcolumntype{F}[1]{>{\hsize=#1\hsize\centering\arraybackslash}X}%
\newcolumntype{G}[1]{>{\hsize=#1\hsize\raggedleft\arraybackslash}X}%
%\newcolumntype{C}[2]{>{\hsize=#1\hsize\columncolor{#2}\centering\arraybackslash}X}%

\usepackage{indentfirst}
\usepackage{setspace}
\onehalfspacing

\usepackage{enumitem}
\setlist{nolistsep}

% % % % % % % % % % % % % % % % % % %
% Нумерация абзацев
% % % % % % % % % % % % % % % % % % %

\renewcommand{\theenumi}{\arabic{enumi}}% Меняем везде перечисления на цифра.цифра
%\renewcommand{\labelenumi}{\arabic{enumi}}% Меняем везде перечисления на цифра.цифра
\renewcommand{\theenumii}{.\arabic{enumii}}% Меняем везде перечисления на цифра.цифра
\renewcommand{\labelenumii}{\arabic{enumi}.\arabic{enumii}.}% Меняем везде перечисления на цифра.цифра
\renewcommand{\theenumiii}{.\arabic{enumiii}}% Меняем везде перечисления на цифра.цифра
\renewcommand{\labelenumiii}{\arabic{enumi}.\arabic{enumii}.\arabic{enumiii}.}% Меняем везде перечисления на цифра.цифра

\renewcommand{\thesection}{\arabic{section}}
%\renewcommand{\thesubsection}{\arabic{subsection}}
%\renewcommand{\thesubsubsection}{\arabic{subsubsection}}
\setlength{\intextsep}{1.75\intextsep}
\setcounter{secnumdepth}{5}
\setcounter{tocdepth}{5}
%\addtolength{\parskip}{8pt}


% % % % % % % % % % % % % % % % % % %
% Новые команды
% % % % % % % % % % % % % % % % % % %

\newcommand{\off}[1]{{\color{gray}#1}}

%\newcommand{\todo}[1]{\marginpar{\color{red} \footnotesize #1}}
%\newcommand{\todonote}[1]{\todo{#1}}
%\newcommand{\toask}[1]{\marginpar{\color{blue} \footnotesize #1}}

%\newcommand{\todo}[1]{}
%\newcommand{\todonote}[1]{{\color{red}#1}}
%\newcommand{\toask}[1]{{\color{blue}#1}}

\newcommand{\todo}[1]{}
\newcommand{\todonote}[1]{}
\newcommand{\toask}[1]{}


\newcommand{\comm}[1]{\marginpar{\color{red}\Large \textbf{!!}}}
\newcommand{\alt}[2]{$\overset{\text{#2}}{\text{#1}}$}
\newcommand{\tweakedsim}{{\raise.17ex\hbox{$\scriptstyle\sim$}}}
\newcommand{\ceil}[1]{\left\lceil#1\right\rceil}
\newcommand{\abs}[1]{\left|#1\right|}
\renewcommand{\phi}{\varphi}

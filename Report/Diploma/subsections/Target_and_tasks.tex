Главной целью данной работы является разработка программного обеспечения для системы на кристалле Zynq-7000 и доведение стенда до готовности к предстоящей эксплуатации по назначению. Таким образом, предстоит довольно широкий ряд работ, от разработки дизайна программируемой логики до конфигурации операционной системы и написания веб-сервера для взаимодействия со стендом.\par
Ранее для СнК уже проводились следующие работы:
\begin{itemize}
    \item создание сервера и клиентской части, а также процессорной системы для их тестирования --- необходима доработка;
    \item конфигурация операционной системы --- остался только бинарный файл и тот под другую версию кристалла, который установлен на тестовой плате;
    \item разработка программируемой логики --- некоторые модули системы завершены, некоторые требуют доработки или написания с нуля.
\end{itemize}\par
Стоит отметить, что данные модули были реализованы отдельно друг от друга без возможности совместного функционирования.\par
Итак, в рамках данной работы были поставлены следующие задачи:
\begin{itemize}
    \item доработка ранее написанной программируемой логики, проектирование процессорной системы и их интеграция для совместной работы;
    \item разработка дизайна программируемой логики для подсчёта статистики данных с АЦП;
    \item конфигурация операционной системы;
    \item доработка сервера и клиентского веб-интерфейса, расширение его функционала.
\end{itemize}


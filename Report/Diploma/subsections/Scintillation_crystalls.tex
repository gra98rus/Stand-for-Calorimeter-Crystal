Как уже было сказано ранее, сцинтилляторы --- это вещества, способные излучать свет при поглощении ионизирующего излучения. Сцинтилляторы характеризуются множеством параметров, но основными являются:
\begin{itemize}
    \item конверсионная эффективность;
    \item технический выход;
    \item время высвечивания.
\end{itemize}\par
Конверсионной эффективностью или физическим выходом называется отношение энергии световой вспышки к энергии, потерянной частицей в кристалле. Таким образом, физический выход характеризует эффективность преобразования энергии ионизирующей частицы в световую в сцинтилляторе. Как правило, данная характеристика лежит в диапазоне от долей процента до десятков процентов.\par
Однако высокое значение конверсионной эффективности не является показателем пригодности вещества в качестве сцинтиллятора в детекторе. Для его использования необходимо, чтобы излучаемый свет мог свободно покидать пределы кристалла. Отношение энегрии световой вспышки, вышедшей из кристалла, к полной энегрии, потерянной частицей в нём, называется техническим выходом или технической эффективностью. Именно этот параметр является основопологающим в определении удовлетворительности качества сцинтиллятора. Он зависит от множества аспектов: толщины слоя сцинтиллятора, состояния его поверхности, концентрации поглощающих примесей, прозрачности кристалла к собственному излучению и так далее.\par
Зачастую интенсивность излучения кристалла $I$ в зависимости от времени $t$ описывается экспоненциальной формулой:\par
\begin{equation}
    I(t) = I_0 e^{-{\frac t {\tau}}},
\end{equation}
где $I_0$ - амплитуда светового импульса, $\tau$ - время, в течение которого интенсивность излучения падает в $e$ раз, называется временем высвечивания сцинтиллятора.\par
В настоящей работе время высвечивания кристалла является очень важным параметром, поскольку он определяет время, а вместе с ним и объём памяти, который необходимо выделять для правильной записи экспериментальных данных. Данная деталь будет описана ниже при рассмотрении технической реализации системы.

Серверная часть разработана на языке python с использованием программной платформы Django Framework. Данные инструменты были выбраны ввиду наличия некоторых наработок для настоящей задачи. Среди достоинств такого решения можно выделить:\par
\begin{itemize}
    \item простота добавления интерпретатора языка python в файловую систему, ввиду присутствия его в пакете PetaLinux Tools;
    \item подробная документация;
    \item скорость разработки (к примеру, язык C++ избыточно сложен для веб-разработки);
    \item отсутствие необходимости в кросс-компилляции
    \item расширяемость.
\end{itemize}\par
Главной задачей сервера является реализация отправки пользовательских комманд и параметров в модуль виртуальных регистров и считывание данных из двухпортовой памяти. Для этого были разработаны обработчики HTTP запросов, а также функции для работы с регистрами и чтения данных из памяти. Для доступа к переферии использован пакет python-periphery.\par
